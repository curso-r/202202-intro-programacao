% Options for packages loaded elsewhere
\PassOptionsToPackage{unicode}{hyperref}
\PassOptionsToPackage{hyphens}{url}
%
\documentclass[
]{article}
\title{A força do povo}
\author{Por \emph{Alexandre Santos}}
\date{}

\usepackage{amsmath,amssymb}
\usepackage{lmodern}
\usepackage{iftex}
\ifPDFTeX
  \usepackage[T1]{fontenc}
  \usepackage[utf8]{inputenc}
  \usepackage{textcomp} % provide euro and other symbols
\else % if luatex or xetex
  \usepackage{unicode-math}
  \defaultfontfeatures{Scale=MatchLowercase}
  \defaultfontfeatures[\rmfamily]{Ligatures=TeX,Scale=1}
\fi
% Use upquote if available, for straight quotes in verbatim environments
\IfFileExists{upquote.sty}{\usepackage{upquote}}{}
\IfFileExists{microtype.sty}{% use microtype if available
  \usepackage[]{microtype}
  \UseMicrotypeSet[protrusion]{basicmath} % disable protrusion for tt fonts
}{}
\makeatletter
\@ifundefined{KOMAClassName}{% if non-KOMA class
  \IfFileExists{parskip.sty}{%
    \usepackage{parskip}
  }{% else
    \setlength{\parindent}{0pt}
    \setlength{\parskip}{6pt plus 2pt minus 1pt}}
}{% if KOMA class
  \KOMAoptions{parskip=half}}
\makeatother
\usepackage{xcolor}
\IfFileExists{xurl.sty}{\usepackage{xurl}}{} % add URL line breaks if available
\IfFileExists{bookmark.sty}{\usepackage{bookmark}}{\usepackage{hyperref}}
\hypersetup{
  pdftitle={A força do povo},
  pdfauthor={Por Alexandre Santos},
  hidelinks,
  pdfcreator={LaTeX via pandoc}}
\urlstyle{same} % disable monospaced font for URLs
\usepackage[margin=1in]{geometry}
\usepackage{longtable,booktabs,array}
\usepackage{calc} % for calculating minipage widths
% Correct order of tables after \paragraph or \subparagraph
\usepackage{etoolbox}
\makeatletter
\patchcmd\longtable{\par}{\if@noskipsec\mbox{}\fi\par}{}{}
\makeatother
% Allow footnotes in longtable head/foot
\IfFileExists{footnotehyper.sty}{\usepackage{footnotehyper}}{\usepackage{footnote}}
\makesavenoteenv{longtable}
\usepackage{graphicx}
\makeatletter
\def\maxwidth{\ifdim\Gin@nat@width>\linewidth\linewidth\else\Gin@nat@width\fi}
\def\maxheight{\ifdim\Gin@nat@height>\textheight\textheight\else\Gin@nat@height\fi}
\makeatother
% Scale images if necessary, so that they will not overflow the page
% margins by default, and it is still possible to overwrite the defaults
% using explicit options in \includegraphics[width, height, ...]{}
\setkeys{Gin}{width=\maxwidth,height=\maxheight,keepaspectratio}
% Set default figure placement to htbp
\makeatletter
\def\fps@figure{htbp}
\makeatother
\setlength{\emergencystretch}{3em} % prevent overfull lines
\providecommand{\tightlist}{%
  \setlength{\itemsep}{0pt}\setlength{\parskip}{0pt}}
\setcounter{secnumdepth}{-\maxdimen} % remove section numbering
\ifLuaTeX
  \usepackage{selnolig}  % disable illegal ligatures
\fi

\begin{document}
\maketitle

{
\setcounter{tocdepth}{2}
\tableofcontents
}
\hypertarget{sport-club-corinthians-paulista}{%
\subsection{Sport Club Corinthians
Paulista}\label{sport-club-corinthians-paulista}}

\includegraphics{https://publisher-publish.s3.eu-central-1.amazonaws.com/pb-brasil247/swp/jtjeq9/media/20190521010540_d2cc7e852b252172288f57d0e2a8d0607ddbb26bcbaf4c9ba074e1cc4a5ab580.jpeg}
\textbf{Legenda}: escudo atual do Sport Club Corinthians Paulista

Fundado em 1° de Setembro de 1910 por operários, traz na sua fundação a
marca impressa na alma de ser um clube de representatividade popular, do
povo.\\
Time de uma torcida apaixonada, sim, time de uma torcida, apaixonada,
louca, ou simplesmente um bando de loucos.

A força desta torcida foi testada em outros momentos históricos como na
Invasão Corinthiana

\includegraphics{https://static.corinthians.com.br/uploads/1606999201e4873aa9a05cc5ed839561d121516766.jpg}
\textbf{Legenda}: Invasão corinthiana no Maracanã em 76

Esta paixão venceu a seca de titulos de 23 anos, de maneira sofrida em
1977, a seca transformou um time em religião, uma nação.

\includegraphics{http://cdn.espn.com.br/image/wide/622_4e760d67-01e5-3328-b629-f9628024d81b.png}\\
\textbf{Legenda}: fim do jejum em 77, gol de Basílio.

Inumeros momentos demosntraram a relevância desta torcida para o
rendimento esportivo do corinthians, por exemplo, no 2° título mundial
no Japão.

\includegraphics{https://static-wp-tor15-prd.torcedores.com/wp-content/uploads/2020/03/corinthians-10-829x397.jpg}
\textbf{Legenda}: Mundial de clubes do Japão

\hypertarget{importuxe2ncia-da-fiel-nos-resultados-esportivos-do-futebol-profissional}{%
\subsection{Importância da Fiel nos resultados esportivos do futebol
profissional}\label{importuxe2ncia-da-fiel-nos-resultados-esportivos-do-futebol-profissional}}

É indiscutivel a importância da torcida nas vitórias deste clube,
entretanto, uma análise detalhada da importância do mando de campo no
campeonato brasileiro no formato de pontos corridos, ou seja, de 2003
até 2021, traria uma visão quantitativa do seu impacto.\\
Portanto, o objetivo desta análise e descrever o importancia do mando de
campo nos resultados, ou seja, pontuação do Corinthians em Campeonato
Brasileiro (2003-2021).

\hypertarget{descriuxe7uxe3o-dos-resultados-do-corinthians-na-era-dos-pontos-corridos.}{%
\subsubsection{1. Descrição dos resultados do Corinthians na era dos
pontos
corridos.}\label{descriuxe7uxe3o-dos-resultados-do-corinthians-na-era-dos-pontos-corridos.}}

\hypertarget{tabela-1.-descriuxe7uxe3o-dos-resultados-do-corinthians-no-campeonato-brasileiro-2003-2021.}{%
\subparagraph{Tabela 1. Descrição dos resultados do Corinthians no
Campeonato Brasileiro
(2003-2021).}\label{tabela-1.-descriuxe7uxe3o-dos-resultados-do-corinthians-no-campeonato-brasileiro-2003-2021.}}

\begin{longtable}[]{@{}lcc@{}}
\toprule
& \textbf{Mandante} & \textbf{Visitante} \\
\midrule
\endhead
\textbf{Número de Jogos} & 352 & 352 \\
\textbf{Número de vitórias} & 193 & 103 \\
\textbf{Número de derrotas} & 64 & 140 \\
\textbf{Números de empates} & 95 & 109 \\
\textbf{Pontos disputados} & 1026 & 1026 \\
\textbf{Pontos conquistados} & 674 & 418 \\
\textbf{Pontos conquistados(\%)} & 65,7 & 40,7 \\
\bottomrule
\end{longtable}

\hypertarget{figura-1.-pontuauxe7uxe3o-corinthians-por-ano}{%
\subparagraph{Figura 1. Pontuação Corinthians por
ano}\label{figura-1.-pontuauxe7uxe3o-corinthians-por-ano}}

\includegraphics{arena_files/figure-latex/unnamed-chunk-68-1.pdf}

\hypertarget{figura-2.-porcentagem-de-pontos-do-corinthians}{%
\subparagraph{Figura 2. Porcentagem de pontos do
Corinthians}\label{figura-2.-porcentagem-de-pontos-do-corinthians}}

\includegraphics{arena_files/figure-latex/unnamed-chunk-69-1.pdf}

\hypertarget{fazer-tabela-saldo-de-gols-anos-e-campeonatos}{%
\section{fazer tabela saldo de gols anos e
campeonatos}\label{fazer-tabela-saldo-de-gols-anos-e-campeonatos}}

\#\url{https://tablesgenerator.com/markdown_tables\#}

\end{document}
